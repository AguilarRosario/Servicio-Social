\documentclass[10pt,letterpaper,spanish,twoside]{report}

\usepackage{practica}
\newcommand{\docdate}{
  \vspace{2em}
   \begin{flushright}
     Ciudad de México. \datedayname~\today.
   \end{flushright}
  \vspace{2em}
}

\begin{document}
\docdate

\begin{center}
\textsc{\asignatura}\vspace{.2em}
\end{center}

\textsc{Práctica VIII. Aplicación de filtros FIR pasa banda para detección de P300 en EEG}

\textsc{Objetivo:}texto

\textsc{Actividades:}
\begin{enumerate}
  \item Descargar el archivo *link*
  \item Describir la señal de EEG
  \item Diseñe un filtro FIR para la detección de P300
  \item Implemente el filtro con fase cero
  \item Calcule el promedio coherente de la señal original y filtrada
  \item Compare las latencias del P300 entre los promedios coherentes  
\end{enumerate}

\textsc{Entregables}
\begin{itemize}
  \item Bitácora por equipo.
  \item Caracterización de los filtros.
  \item FFT de señal filtrada.
  \item Comparación de latencias de P300.
\end{itemize}

%\textsc{Nota}
%\vspace{2em}
\vfill
\begin{flushright}
\textsc{Elaboró:\\
Ma. del Rosario Aguilar Cruz\\
Enrique Mena Camilo}
\end{flushright}

\end{document}