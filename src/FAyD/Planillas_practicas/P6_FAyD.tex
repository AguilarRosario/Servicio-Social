\documentclass[10pt,letterpaper,spanish,twoside]{report}

\usepackage{practica}
\newcommand{\docdate}{
  \vspace{2em}
   \begin{flushright}
     Ciudad de México. \datedayname~\today.
   \end{flushright}
  \vspace{2em}
}

\begin{document}
\docdate

\begin{center}
 \textsc{\asignatura}\vspace{.2em}
\end{center}

\textsc{Práctica VI. Aplicación de filtros chebyshev digitales tipo 1 y 2 rechaza banda}

\textsc{Objetivo:} 
Aplicar los conocimientos adquiridos sobre el diseño de filtros Chebyshev digitales para el procesamiento de señales biomédicas

\textsc{Actividades}
\begin{enumerate}
  \item Descargar el archivo de audio 'PCG$\_$60.npz' https://goo.gl/xCkryR
  \item Obtener la FFT de la señal de forma digital
  \item Identificar la interferencia en la FFT
  \item Diseñar un filtro digital tipo FIR rechaza banda orden 4 tipo 1 para eliminar la interferencia
  \item Implementar el filtro con fase cero en código
  \item Caracterizar la respuesta en frecuencia del filtro
  \item Diseñar un filtro digital tipo FIR rechaza banda orden 4 tipo 2 para eliminar la interferencia
  \item Implementar el filtro con fase cero en código
  \item Caracterizar la respuesta en frecuencia del filtro
\end{enumerate}

\textsc{Entregables}
\begin{itemize}
  \item Bitácora por equipo.
  \item Caracterización de los filtros.
  \item Contenido espectral de señales originales y filtradas.
\end{itemize}

%\textsc{Nota: }
%\vspace{2em}
\vfill
\begin{flushright}
\textsc{Elaboró:\\
Ma. del Rosario Aguilar Cruz\\
Enrique Mena Camilo}
\end{flushright}

\end{document}