\documentclass[10pt,letterpaper,spanish,twoside]{report}

\usepackage{practica}
\newcommand{\docdate}{
  \vspace{2em}
   \begin{flushright}
     Ciudad de México. \datedayname~\today.
   \end{flushright}
  \vspace{2em}
}

\begin{document}
\docdate

\begin{center}
 \textsc{\asignatura}\vspace{.2em}
\end{center}

\textsc{Práctica V. Aplicación de filtros Chebyshev tipo 1 y 2 digitales para detección de actividad mental en EEG}

\textsc{Objetivo:}texto

\textsc{Actividades:}
\begin{enumerate}
  \item Descargar el archivo *link*
  \item Diseñar filtros Chebyshev tipo 1 para segmentar el EEG en sus cuatro bandas principales (alfa, beta, theta y delta)
  \item Implementar dichos filtros con fase cero
  \item Comprobar la segmentación de las cuatro bandas por medio de su FFT
  \item Caracterizar la respuesta en frecuencia de los filtros
  \item Diseñar filtros Chebyshev tipo 2 para segmentar el EEG en sus cuatro bandas principales (alfa, beta, theta y delta)
  \item Implementar dichos filtros con fase cero
  \item Comprobar la segmentación de las cuatro bandas por medio de su FFT
  \item Caracterizar la respuesta en frecuencia de los filtros
  \item Identificación de la actividad mental de interés en las diferentes épocas
\end{enumerate}

\textsc{Entregables}
\begin{itemize}
  \item Bitácora por equipo.
  \item Caracterización de los filtros.
  \item Código.
  \item Identificación de actividad mental.
\end{itemize}

%\textsc{Nota}
%\vspace{2em}
\vfill
\begin{flushright}
\textsc{Elaboró:\\
Ma. del Rosario Aguilar Cruz\\
Enrique Mena Camilo}
\end{flushright}

\end{document}