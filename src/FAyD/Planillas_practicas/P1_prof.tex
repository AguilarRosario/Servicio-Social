\documentclass[10pt,letterpaper,spanish,twoside]{report}

\usepackage{practica}
\newcommand{\docdate}{
  \vspace{2em}
   \begin{flushright}
     Ciudad de México. \datedayname~\today.
   \end{flushright}
  \vspace{2em}
}

\begin{document}
\docdate

\begin{center}
 \textsc{\asignatura}\vspace{.2em}
\end{center}

\textsc{Práctica 1. Aplicación de filtros analógicos tipo Butterworth pasa bajas a señales biomédicas reales}

\textsc{Objetivo:} 
Aplicar los conocimientos adquiridos sobre el diseño de filtros analógicos Chebyshev tipo 1 para el procesamiento de señales biomédicas reales adquiridas simulando su adquisición en tiempo real.

\textsc{Actividades}
\begin{enumerate}
  \item Descargar señal de presión arterial del banco de señales de physioNet.org con una duración de un minuto
  \item La señal tiene una frecuencia de muestreo de 100 Hz, por lo tanto, fue necesario remuestrear la señal a 1000 Hz y este proceso se puede consultar en el archivo de Señales$\_$Utilizadas.ipynb encontrado en la siguiente liga *
  \item Contaminar señal con ruido de alta frecuencia 
  \item Grabar señal en archivos .wav
  \item 
\end{enumerate}

\textsc{Entregables}
\begin{itemize}
  \item Bitácora por equipo
  \item Caracterización del filtro
  \item Circuito funcionando
  \item Contenido espectral de la señal original y la señal filtrada
\end{itemize}


%\textsc{Nota}
%\vspace{2em}
\vfill
\begin{flushright}
\textsc{Elaboró:\\
Ma. del Rosario Aguilar Cruz\\
Enrique Mena Camilo}
\end{flushright}

\end{document}