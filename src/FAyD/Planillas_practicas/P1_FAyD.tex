\documentclass[10pt,letterpaper,spanish,twoside]{report}

\usepackage{practica}
\newcommand{\docdate}{
  \vspace{2em}
   \begin{flushright}
     Ciudad de México. \datedayname~\today.
   \end{flushright}
  \vspace{2em}
}

\begin{document}
\docdate

\begin{center}
 \textsc{\asignatura}\vspace{.2em}
\end{center}

\textsc{Práctica I. Aplicación de filtros analógicos tipo Butterworth pasa bajas a señales biomédicas reales}

\textsc{Objetivo:} Aplicar los conocimientos sobre el diseño de filtros Butterworth pasa bajas para procesamiento de señales biomédicas reales simulando su adquisición en tiempo real

\textsc{Actividades}
\begin{enumerate}
  \item Descargar el archivo 'BPW$\_$noise.wav' goo.gl/8TvoDM
  \item Obtener la FFT de la señal con el osciloscopio.
  \item Identificar ruido de la señal en la FFT.
  \item Diseñar filtro pasa bajas orden 2 tipo Butterworth para filtrar la señal.
  \item Implementar el diseño con una topología Sallen-Key.
  \item Caracterizar respuesta en frecuencia del filtro.
  \item Diseñar filtro pasa bajas orden 4 tipo Butterworth para filtrar la señal.
  \item Implementar el diseño con una topología Sallen-Key.
  \item Caracterizar respuesta en frecuencia del filtro 
\end{enumerate}

\textsc{Entregables}
\begin{itemize}
  \item Bitácora por equipo.
  \item Caracterización de los filtros.
  \item Circuitos funcionando.
  \item Contenido espectral de señales originales y filtradas.
\end{itemize}

\textsc{Nota: La caracterización de los filtros debe incluir magnitud y fase con 15 mediciones por década y 10 mediciones alrededor de la frecuencia de corte.}
%\vspace{2em}
\vfill
\begin{flushright}
\textsc{Elaboró:\\
Ma. del Rosario Aguilar Cruz\\
Enrique Mena Camilo}
\end{flushright}

\end{document}