\documentclass[10pt,letterpaper,spanish,twoside]{report}

\usepackage{practica}
\newcommand{\docdate}{
  \vspace{2em}
   \begin{flushright}
     Ciudad de México. \datedayname~\today.
   \end{flushright}
  \vspace{2em}
}

\begin{document}
\docdate

\begin{center}
 \textsc{\asignatura}\vspace{.2em}
\end{center}

\textsc{Práctica II. Aplicación de filtros analógicos tipo Butterworth pasa altas a señales biomédicas reales}

\textsc{Objetivo:}Aplicar los conocimientos sobre el diseño de filtros Butterworth pasa bajas para procesamiento de señales biomédicas reales simulando su adquisición en tiempo real

\textsc{Actividades:}
\begin{enumerate}
  \item Descargar el archivo de audio 'ECG$\_$resp.wav' https://goo.gl/gKKAfd
  \item Obtener la FFT de la señal en el osciloscopio
  \item Identificar la interferencia en la FFT
  \item Diseñar un filtro analógico tipo Butterworth pasa altas orden 2 para eliminar la interferencia
  \item Implementar el filtro con topología Sallen-Key
  \item Caracterizar la respuesta en frecuencia del filtro
  \item Diseñar un filtro analógico tipo Butterworth pasa altas orden 4 para eliminar la interferencia
  \item Implementar el filtro con topología Sallen-Key
  \item Caracterizar la respuesta en frecuencia del filtro
\end{enumerate}

\textsc{Entregables}
\begin{itemize}
  \item Bitácora por equipo.
  \item Caracterización de los filtros.
  \item Circuitos funcionando.
  \item Contenido espectral de señales originales y filtradas.
\end{itemize}


%\textsc{Nota}
%\vspace{2em}
\vfill
\begin{flushright}
\textsc{Elaboró:\\
Ma. del Rosario Aguilar Cruz\\
Enrique Mena Camilo}
\end{flushright}

\end{document}