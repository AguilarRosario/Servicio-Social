\documentclass[10pt,letterpaper,spanish,twoside]{report}

\usepackage{practica}
\newcommand{\docdate}{
  \vspace{2em}
   \begin{flushright}
     Ciudad de México. \datedayname~\today.
   \end{flushright}
  \vspace{2em}
}

\begin{document}
\docdate

\begin{center}
 \textsc{\asignatura}\vspace{.2em}
\end{center}

\textsc{Práctica 3. Aplicación de filtros digitales tipo Butterworth rechaza banda a señales biomédicas reales}

\textsc{Objetivo:}Aplicar los conocimientos sobre el diseño de filtros Butterworth rechaza banda para procesamiento de señales biomédicas reales simulando su adquisición en tiempo real

\textsc{Actividades:}
\begin{enumerate}
  \item Descargar el archivo de audio 'PCG$\_$noise.wav' https://goo.gl/xCkryR
  \item Obtener la FFT de la señal de forma digital
  \item Identificar la interferencia en la FFT
  \item Diseñar un filtro digital tipo Butterworth rechaza banda orden 2 para eliminar la interferencia
  \item Implementar el filtro con fase cero en código
  \item Caracterizar la respuesta en frecuencia del filtro
  \item Diseñar un filtro digital tipo Butterworth rechaza banda orden 4 para eliminar la interferencia
  \item Implementar el filtro con fase cero en código
  \item Caracterizar la respuesta en frecuencia del filtro
\end{enumerate}

\textsc{Entregables}
\begin{itemize}
  \item Bitácora por equipo.
  \item Caracterización de los filtros.
  \item Código.
  \item Comparación entre audio contaminado y filtrado.
\end{itemize}


%\textsc{Nota}
%\vspace{2em}
\vfill
\begin{flushright}
\textsc{Elaboró:\\
Ma. del Rosario Aguilar Cruz\\
Enrique Mena Camilo}
\end{flushright}

\end{document}