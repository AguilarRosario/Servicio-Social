\documentclass[10pt,letterpaper,spanish,twoside]{report}

\usepackage{practica}
\newcommand{\docdate}{
  \vspace{2em}
   \begin{flushright}
     Ciudad de México. \datedayname~\today.
   \end{flushright}
  \vspace{2em}
}

\begin{document}
\docdate

\begin{center}
 \textsc{\asignatura}\vspace{.2em}
\end{center}

\textsc{Práctica 0. Diseño de sistema mínimo para la simulación de adquisición de señales bioeléctricas en tiempo real}

\textsc{Objetivo:} Construir un sistema preamplificador cuya entrada sea la señal de audio analógica generada por un dispositivo electrónico, por ejemplo: computadora, teléfono móvil, tablet, etc., dicho dispositivo reproducirá una pista de audio en formato .wav, la cual contiene información de una señal de origen fisiológico, en otras palabras, se estará simulando la adquisición de señal de paciente en tiempo real.

\textsc{Actividades}
\begin{enumerate}
  \item Acople un conector de 3.5 mm con caimanes para simular la adquisición de la señal reproducida por el dispositivo móvil.
  \item Diseñe un amplificador de instrumentación con ganancia variable tal que la salida pueda estar entre $\pm$ 1.5 V y $\pm$ 5.5 V.
  \item Caracterizar la respuesta en frecuencia del preamplificador con 15 mediciones por década y 10 mediciones alrededor de la frecuencia de corte.
  \item Evalúe el funcionamiento del circuito con 3 señales de origen biomédico
  \begin{itemize}
  	\item Electrocardiograma https://goo.gl/7XfVcG
  	\item Onda de presión arterial https://goo.gl/RVKzue
  	\item Señal de respiración obtenida mediante  impedancimetría http://drives.news/google594
  \end{itemize}
\end{enumerate}

\textsc{Entregables}
\begin{itemize}
  \item Bitácora por equipo
  \item Circuito funcionando
  \item Contenido espectral de las señales fisiológicas
\end{itemize}


%\textsc{Nota}
%\vspace{2em}
\vfill
\begin{flushright}
\textsc{Elaboró:\\
Ma. del Rosario Aguilar Cruz\\
Enrique Mena Camilo}
\end{flushright}

\end{document}