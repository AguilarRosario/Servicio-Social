
\newcommand{\escuela}{Universidad Autónoma Metropolitana Unidad Iztapalapa}
\newcommand{\departamento}{DIE}
\newcommand{\licenciaturas}{Ingeniería Biomédica\xspace}
\newcommand{\periodo}{19-I }
\newcommand{\asignatura}{Filtrado Analógico y Digital\xspace}
\newcommand{\clave}{2151020}
\newcommand{\profesor}{Omar Piña Ramírez}
%\newcommand{\sigla}{IE063} %aplica para Ibero


%horario
\newcommand{\Lunes}    { 15:00-18:00 h\\ AT109\\Laboratorio  }
\newcommand{\Martes}   { 16:30-18:00 h\\ C121\\Teoría               }
\newcommand{\Miercoles}{                                          }
\newcommand{\Jueves}   { 16:30-18:00 h\\ C121\\Teoría               }
\newcommand{\Viernes}  {                                          }

%contacto
\newcommand{\contacto}{
\begin{tabular}{rl}
	  Profesor: &\profesor \\
	  cubículo: &T227\\
	  e-mail: &delozath.edu@gmail.com \\
	  web: &https://classroom.google.com\\
	   & clave: c83o345\\
	\end{tabular}
}

%sobre el curso
\newcommand{\importancia}{La UEA \asignatura es un curso que integra los conocimientos de cursos previos como son: Señales y Sistemas I y II, Introducción a la Programación, Circuitos Electrónicos I y II, Álgebra Lineal I y II. Dada esta naturaleza, el curso de filtrado brindará a los alumnos la oportunidad de resolver problemas del mundo real en el contexto de la ingeniería biomédica haciendo uso de los conocimientos que hasta el momento hayan acumulado a lo largo de su licenciatura.}

\newcommand{\objetivo}{La el objetivo de la UEA \asignatura es proporcionar al alumno los fundamentos teórico-prácticos del filtrado en los dominios analógico y digital, además de proporcionar una panorámica de su potencial uso en el ámbito laboral y sus generalizaciones para el ámbito académico y de investigación.}


\newcommand{\objetivos}{
  \begin{itemize}
    \item Analizar problemas que involucren el diseño e implementación de filtros analógicos y digitales
    \item Implementar el mejor esquema de filtrado dadas las características del problema a resolver
    \item Conceptualizar y llevar a cabo la solución de problemas mediante el cambio de domino y la integración de conocimiento
  \end{itemize}
}

\newcommand{\crequeridos}{
	\begin{itemize}
 	 \item Señales y Sistemas I y II
 	 \item Introducción a la Programación
 	 \item Circuitos Electrónicos I y II
 	 \item Álgebra Lineal I y II
 	 \item Ecuaciones Diferenciales Ordinarias
	\end{itemize}
}

\newcommand{\copcionales}{
	\begin{itemize}
 	 \item Python
 	 \item Variable Compleja
 	 \item Electrónica de Potencia
 	 \item Programación Avanzada
 	 \item Métodos Numéricos
	\end{itemize}
}

\newcommand{\firma}{
\begin{center}
  \begin{tabular}{c}
    \hline\\
    Dr. Omar Piña Ramírez\\
    Profesor Asociado D\\
    Laboratorio de Investigación en Neuroimagenología\\
    Departamento de Ingeniería Eléctrica\\
  \end{tabular}
\end{center}
}