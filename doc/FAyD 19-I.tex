% \TeX spellcheck = es_MX
\documentclass[letterpaper, 11pt]{article}

\usepackage{planeacion}

\begin{document}

%TODO pagina de inicio en un archivo tex a parte
\thispagestyle{plain}

\begin{minipage}{0.3\linewidth}
	\vspace*{-60pt}
	\includegraphics[width=\linewidth]{logo}  
\end{minipage}~~
\begin{minipage}{0.8\linewidth}
	\vspace*{-20pt}
	{\Large \licenciaturas\\}
	{\large \periodo\\ 
		\asignatura}, {\clave}\\

	\contacto
\end{minipage}
\hrule\hrule

\vspace{10pt}
	Horario: 
\begin{center}
{\setlength{\extrarowheight}{14pt}%
\begin{tabular}{lp{.17\linewidth}p{.17\linewidth}p{.17\linewidth}p{.17\linewidth}p{.17\linewidth}}
 &   \multicolumn{1}{c}{Lunes}
   & \multicolumn{1}{c}{Martes}
   & \multicolumn{1}{c}{Miércoles}
   & \multicolumn{1}{c}{Jueves}
   & \multicolumn{1}{c}{Viernes} \\
\renewcommand{\arraystretch}{2}%
 &   \cellcolor{MainColor!70} \parbox{.17\textwidth}{\centering \Lunes}
   & \cellcolor{MainColor}    \parbox{.17\textwidth}{\centering \Martes}
   & \cellcolor{MainColor!20} \parbox{.17\textwidth}{\centering \Miercoles}
   & \cellcolor{MainColor}    \parbox{.17\textwidth}{\centering \Jueves}
   & \cellcolor{MainColor!20} \parbox{.17\textwidth}{\centering \Viernes} \\[4ex]
\end{tabular} }\\[2em]
\end{center}

Asesorías:\\
Solicitar horario por correo electrónico. 

\vspace* {5pt}
\hrule
\section{Importancia del curso}
 \importancia

\vspace* {5pt}
\hrule
%\vspace* {2pt}
\section{Objetivos}
\objetivo
%\subsection{Objetivos generales}


\subsection{Objetivos específicos}
El alumno será capaz de:
\objetivos

\vspace* {5pt}
\hrule

\section{Conocimientos Previos}

\subsection{Esenciales}
\crequeridos

\subsection{Recomendables}
\copcionales



\vspace* {5pt}
\hrule




\section{Temario General}


\renewcommand{\labelenumi}{\arabic{enumi}.}
\renewcommand{\labelenumii}{\labelenumi\arabic{enumii}.}
\renewcommand{\labelenumiii}{\labelenumii\arabic{enumiii}. }
\begin{enumerate}
  \item{ Introducción
          \begin{enumerate}
            \item{ Conceptos Fundamentales\week
                   \begin{enumerate}
  \item Presentación de la UEA
  \item Números complejos origen y periodicidad
  \item Identidad de Euler
  \item Números complejos para representar rotaciones de vectores
  \item Función de transferencia
  \item Función de transferencia como una función de números complejos
  \item Respuesta en frecuencia
  \item Espectro de frecuencias de una señal
\end{enumerate} 
              }
            \item{ Sistemas lineales como filtros\weekrep
                   \begin{enumerate}
 \item Sistemas lineales invariantes en el tiempo (LTI)
 \item  LTI en tiempo continúo y discreto
 \item  Respuesta en frecuencia. El problema del filtrado,  
 \item Filtros ideales, Clasificación de sistemas de filtrado. Filtros pasabajas, (PB), filtros pasa-altas (PA), filtros pasabanda (PBA), filtros de rechazo de banda (RB), filtros all-pass (AP).
 \item Propiedades de las funciones de transferencia características para cada tipo de filtro
 \item Forma general de la función del sistema de cada uno de estos tipos de filtros: Filtros de primer y segundo orden en tiempo continuo
\end{enumerate} 
              }
          \end{enumerate} 
          
       }
  \item{ Filtros analógicos y digitales con aproximación Chebyshev
          \begin{enumerate}
            \item{ Introducción a filtros Chebyshev\weeks
                   \begin{enumerate}
  \item Análisis del espectro de frecuencias de señales bioeléctricas, Aproximación Chebyshev: Ecuación diferencial de Chebyshev, Polinomios de Chebyshev, Función de transferencia de Chebyshev Tipo I en tiempo continuo
  \item Práctica I. Construcción de un sistema mínimo para simular señales bioeléctricas humanas, 
  \item Diseño de fitros analógicos Chebyshev Tipo I, Función de transferencia de un filtro Chebyshev Tipo I pasa bajas en tiempo continuo; topología Sallen-Key de un filtro pasa bajas
  \item Ejercicios
  \item Práctica II. Eliminar ruido de parpadeo (EMG) de los canales frontales de un EEG 
\end{enumerate} 
              }
            \item{ Transformaciones en frecuencia y de dominio\weekss
                   \begin{enumerate}
  \item Transformación de frecuencias: modificar frecuencias de corte. Recapitulación, Análisis y Ejercicios, Función de transferencia de un filtro Chebyshev Tipo I pasa altas, pasa banda y rechaza banda en tiempo continuo; topología Sallen-Key
  \item Diseño de filtros analógicos para aplicaciones en electrofisiología (pasa altas), 
  \item Función de transferencia Chevyshev Tipo II en tiempo continuo. Implementación de filtros Chebyshev con Topología Sallen-Key y State Variable, 
  \item Transformación bilineal: Transformación de M\"{o}bius, Mapeo del espacio S al espacio Z: Transformación bilineal, análisis del error de mapeo. Diseño de filtros digitales pasa bajas a partir de sus equivalentes en tiempo continuo
  \item Implementación de filtros digitales 
  \item Ejercicios y aplicaciones
  \item Examen escrito I
\end{enumerate} 
              }
          \end{enumerate} 
          
       }
       
  \item{ Filtros analógicos y digitales con aproximación Butterworth
          \begin{enumerate}
            \item{ Filtros Butterworth analógicos\weeks
                   \begin{enumerate}
 \item Función de transferencia de un filtro Butterworth pasa bajas en tiempo continuo; topología Sallen-Key de un filtro pasa bajas, 
  \item Práctica III. Eliminar ruido de baja frecuencia de un EKG, 
  \item Ejercicios de transformación en frecuencias para filtros con aproximación Butterworth, Diseño de filtro Butterworth pasa bajas en tiempo continuo
\end{enumerate} 
              }
            \item{ Filtros Butterworth digitales\week
                   \begin{enumerate}
  \item Diseño de filtros Butterworth digitales, 
  \item Ejercicios, 
  \item Práctica IV. Procesamiento de potenciales de membrana celular, 
  \item Diseño de filtros con herramientas de cómputo, 
  \item Examen escrito II, 
  \item Práctica V. Separación de fuentes de señales EKG + EMG + ruido,
\end{enumerate} 
              }
          \end{enumerate} 
          
       }
 
  \item{ Filtros de respuesta finita al impulso (FIR)\weeks
          \begin{enumerate}
            \item{ Diseño e implentación de filtros FIR
                   \begin{enumerate}
  \item Filtros FIR: Características de los filtros no ideales; Función Sync, Especificación de filtros FIR, Características y tipos de filtros FIR, Diseño de filtros FIR con truncamiento de la función Sync (Ventana Cuadrada) , 
  \item Práctica VI. Filtros FIR para extracción de P300, 
  \item Diseño e implementación de filtros FIR con técnica de ventanas, 
  \item Diseño de filtros FIR con truncamiento de la función Sync con Ventanas Hanning y Chebyshev, 
  \item Proyecto: Diseño de filtros analógicos para aplicaciones en respuesta galvánica de la piel (pasa bajas), 
  \item Examen escrito III
  \item Examen global\week
  \item Entrega de proyecto y evaluación final
\end{enumerate} 
              }

          \end{enumerate}        
       }
 
\end{enumerate}


\nocite{Kamal2009,Parhami2007}
\printbibliography[title={Bibliografía Básica}, keyword=basica, sorting=tny, heading=bibnumbered]

\nocite{TivaUG2013,Tiva2014}
\printbibliography[title={Bibliografía Básica}, keyword=complementaria, sorting=tny, heading=bibnumbered]

\hrule


% % % % % % % % % % % % % % % % % % % % % % % % % % % % % % % % 
\vspace* {3pt}
\hrule
\vspace* {2pt}
\section{Evaluación}

Los elementos de evaluación del curso se catalogan en teoría y laboratorio. Para acreditar la materia cada uno de los elementos de dichas categorías debe tener una calificación mínima de 6.0. A continuación se especifican cada uno de los elementos de evaluación del curso.

\subsection{Teoría}  
		\begin{description} 
			\item [Exámenes (50\%):] Se aplicarán cuatro exámenes obligatorios, tres intermedios para evaluaciones parciales y uno al final del curso con carácter de global. Las fechas de aplicación serán las siguientes:
			  \begin{itemize}
			    \item Examen Parcial I. \hspace{5pt}2019.02.21
			    \item Examen Parcial II.\hspace{4pt}2019.03.14
			    \item Examen Parcial III. 2019.04.02
			    \item Examen Global. \hspace{15pt}2019.05.09
			  \end{itemize}
			\item [Tareas, exámenes rápidos y participación (10\%):] Se asignarán tareas y exámenes rápidos acorde con el avance del curso. Estos pueden tener las siguientes calificaciones: \textbf{No entregada} (0 ptos.), \textbf{Incompleta} (1 pto.) y \textbf{Completa} (2 ptos.). 
			
Una tarea se considera incompleta si se realizó menos del 70\% correctamente. De forma similar se realizará con los exámenes rápidos. En el caso de una tarea no entregada se considera cuando una tarea tiene menos del 50\% de las actividades realizadas correctamente.

En cuanto a las participaciones, se considerará únicamente aquellas que contribuyan a consolidar el conocimiento en la clase. En este rubro se considerarán aquellas tareas opcionales que eventualmente se asignen, la calificación se obtendrá de la misma forma que en tareas y exámenes rápidos. 

La calificación $C_{tr}$ de este rubro se obtendrá:
$$C_{tr}=\dfrac{1}{2N}\sum_{i=1}^N x_i,$$
donde $x_i\in\{0,1,2\}$ corresponde a la calificación del $i$-ésimo elemento del rubro, siendo $N$ el número total de elementos de evaluación del rubro.
		\end{description}
	
	\subsection{Laboratorio}
  	\begin{description}
  	  \item [Prácticas (20\%):] Habrá cinco prácticas que los alumnos deberán resolver cabalmente. En los enunciados de cada una se especificarán los elementos de evaluación, los entregables y sus correspondientes ponderaciones. La calificación de cada práctica se realizará sobre diez puntos. Los equipos de laboratorio se conformarán por tres integrantes inscritos a la UEA.
  	  
  	  La calificación $C_{pr}$ de las prácticas se calculará como el promedio simple de las calificaciones individuales de las prácticas.
  	  
  	  \item [Proyecto (10\%):] El proyecto se asignará en la semana 10 y contemplara el desarrollo completo de una aplicación de filtros en el ámbito de la ingeniería biomédica. Los requerimientos mínimos, entregables y sus respectivas ponderaciones se darán a conocer oportunamente. La calificación del proyecto será sobre diez puntos. %El proyecto final de la materia será elegido por los cada equipo de laboratorio y deberá ser autorizado por el profesor. Una vez que aprobado el proyecto, este no podrá cambiarse por otro. Los requerimientos mínimos, entregables y sus respectivas ponderaciones se darán a conocer oportunamente. La calificación del proyecto será sobre diez puntos.
  	  
  	  \item [Bitácora (5\%).] Es de observancia obligatoria el uso de una bitácora para las prácticas de laboratorio y proyecto final. El formato es libre siempre y cuando se cumplan los siguientes requisitos mínimos:
      \begin{itemize}
        \item Completamente enumerada por hoja
        \item Utilizar por lo menos cinco elementos de simbología, por ejemplo: tareas por hacer, tareas realizadas, para buscar después, etc.
        \item Tener un índice general al inicio de la bitácora
        \item Tener una caratula al inicio de la bitácora en donde se plasmen los datos generales del estudiante
        \item Escrita completamente a pluma
        \item En caso de errores, se tachará con línea simple, en otras palabras, no están permitidos rayones, no corrector
      \end{itemize}
      La calificación de la bitácora, en escala de 0 a 10, será el promedio simple de las calificaciones de las revisiones de dicho documento. La revisión de la bitácora será aleatorio para asegurar que se escriba mientra se realizan las prácticas y no posteriormente a la conclusión de cada una de las mismas.
     \item [Video-bitácora (5\%).] Deberá contener segmentos de grabación del desarrollo de las prácticas de laboratorio y proyecto. Podrá contener, aunque no se limita a: explicaciones, hipótesis, prueba de circuitos, discusiones y reflexiones. La calificación de la vídeo-bitácora será el promedio simple de las calificaciones parciales, todas en escala de 0 a 10.
  	\end{description}

%La bitácora y las video-bitácoras deberán ser realizadas para cada práctica y para el proyecto. La calificación en cada caso será el promedio simple de las calificaciones parciales de las prácticas y proyecto. El esquema de calificaciones será sobre diez.
	\subsection{Nota final}

Sea $C_p$ la calificación obtenida con la correspondiente ponderación de los elementos de teoría y de laboratorio. La nota final $N_f$ se obtendrá de acuerdo a:
	\begin{equation}
	N_f = 
		\begin{cases}
		  \text{NA}, & C_p < 6 \\
		  \text{S},  & 6\leq C_p < 7.6\\
		  \text{B},  & 7.8\leq C_p < 8.8\\
		  \text{MB},  & C_p \geq 8.8\\
		\end{cases}
	\end{equation}
%	\begin{equation}
%	C_f = 
%		\begin{cases}
%		5, & \text{si} \ C_p < 6 \\
%		\lceil C_p \rceil, & \text{si} \ mod(C_p,1) \geq 0.55 \\
%		\lfloor C_p \rfloor, & \text{en otro caso.}  
%		\end{cases}
%	\end{equation}
%Donde la función residuo se denota como $mod()$, mientras que las funciones entero siguiente y entero anterior se denotan como $\lceil \cdot \rceil$ y $\lfloor \cdot \rfloor$, respectivamente.

\vspace* {3pt}
\hrule
\vspace* {2pt}

\section{Consideraciones de evaluación}
\begin{enumerate}
 \item Bajo ninguna circunstancia se aplicarán exámenes de reposición, tampoco se permitirán entregas extemporáneas, ni se dejarán trabajos adicionales a los estipulados en la presente planeación. Por tal motivo, se exhorta a los alumnos que pongan empeño desde el principio del curso, así mismo, también se les pide que expongan oportunamente sus inquietudes ya sea de manera oral u escrita.
 \item Los códigos, bitácoras o en general cualquier entregable que tengan un parecido, ya sea entre compañeros a alguna fuente externa, serán sometidos a un sistema de detección de plagio; en caso que este sea detectado se procederá de acuerdo a la sección de Fraude Académico.
\end{enumerate}

\vspace*{5pt}
\hrule
\section{Fraude Académico}

No se tolerará fraude académico alguno y se seguirá estrictamente la Política Sobre Fraude Académico de la Universidad. Cualquier alumno que sea sorprendido cometiendo fraude será reprobado inmediatamente en el curso y su caso remitido al Consejo Técnico.

Con la política como guía, es indispensable puntualizar algunos elementos específicos para esta clase:

\begin{enumerate}
	\item  En los trabajos que se entreguen por escrito, queda estrictamente prohibido copiar cualquier material (ya sea transcrito de una fuente impresa o copiado y pegado a partir de una fuente electrónica) como parte de los mismos. Los trabajos deberán seguir las reglas de todo trabajo técnico-científico de nivel profesional, con referencias a las fuentes de la información, parafraseo y sólo una cantidad razonable de citas textuales. Esto aplica para cualquier sección de los reportes.

	\item Queda estrictamente prohibido el uso de sitios de Internet como fuente de información para los trabajos del curso a menos que sea información de fabricantes, como hojas de especificaciones. Puede utilizarse Internet para localizar artículos, libros, manuales, hojas de especificaciones y otras fuentes de información serias que estén disponibles a través de ese medio, así como publicaciones electrónicas serias, pero definitivamente no podrá utilizarse material cuya referencia sea el propio sitio de Internet.

	\item Se hace especial énfasis en que no debe utilizarse ningún material desarrollado por alumnos de trimestres anteriores. Esto incluye no sólo trabajos escritos, sino software y hardware para las prácticas de laboratorio.%semestres anteriores. Esto incluye no sólo trabajos escritos, sino software y hardware para las prácticas de laboratorio.

	\item Queda prohibido utilizar software obtenido de manera ilegal y no se permitirá que los alumnos realicen copias ilegales de los programas disponibles en las computadoras de la universidad.
\end{enumerate}

\section{Reglamento interno}
\begin{itemize}
  \item Tolerancia para ingresar al salón de clase o laboratorio es 15 minutos luego de la hora oficial de inicio de clase.
  \item El uso de dispositivos electrónicos como son: celulares, tablets o laptops no están prohibidos siempre y cuando se utilicen para propósitos académicos. El uso de cualquier tecnología de la información para propósitos no académicos o que interfieran con el curso de la clase se hará acreedor a una llamada de atención. Cuando se acumulen tres llamadas de atención se tomarán medidas coercitivas como son la suspensión definitiva del uso de cualquier dispositivo electrónico en clase.
  \item El eje conductor del curso será el respeto y cordialidad entre todos los integrantes del proceso de enseñanza aprendizaje, pero siempre conservando el espíritu crítico y la apertura a la discusión de ideas bien fundamentadas.
  \item El bullying y cualquier acto de discriminación están absolutamente prohibidos en el salón de clase. En cualquier caso se llamará la atención una vez y en caso de reincidencia se notificará al Consejo Técnico.
  \item La entrega extemporánea de cualquier entregable está sujeta a la siguiente ecuación
   \begin{equation}
     \large x_w = x{e}^{-0.01t},
   \end{equation}
   donde $x$ es la calificación sin ponderación del entregable, $t$ el tiempo en horas de retardo en la entrega y $x_w$ la calificación correspondiente luego de la ponderación por retardo. Está ecuación está calculada para asignar 6.9 a una tarea que hipotéticamente se calificó con 10, pero fue entregada 36 horas después de la fecha y hora estipuladas.
\end{itemize}

\section*{Anexo A}
Prácticas de laboratorio
\begin{center}
 \begin{tabular}{rp{0.7\textwidth}}
  Práctica I& Construcción de un sistema mínimo para simular señales bioeléctricas humanas\\
Práctica II& Eliminar ruido de parpadeo (EMG) de los canales frontales de un EEG\\
Práctica III& Eliminar ruido de baja frecuencia de un EKG\\
Práctica IV& Procesamiento de potenciales de membrana celular\\
Práctica V& Separación de fuentes de señales EKG + EMG + ruido\\
Práctica VI& Filtros FIR para extracción de P300\\
Proyecto &Diseño de filtros analógicos y digitales para aplicaciones en respuesta galvánica de la piel
 \end{tabular}
\end{center}

%\newpage
\section*{Anexo B}
Fechas importantes
\begin{center}
  \begin{tabular}{rl}
  	\multicolumn{2}{c}{\bf Exámenes}\\\hline
    Examen I  & 2019.02.21\\
    Examen II & 2019.03.14\\
    Examen III& 2019.04.02\\
    Examen Global& 2019.04.09\\
%    \multicolumn{2}{c}{\bf Entregas}\\\hline
%    Práctica I  & 2019.02.05\\
%    Práctica II & 2019.02.25\\
%    Práctica III& 2019.03.21\\
%    Práctica IV & 2019.04.04\\
%    Práctica V  & 2019.05.14\\
%    Proyecto    & \\
    \hline
  \end{tabular}
\end{center}

\vfill
\firma

\end{document}
